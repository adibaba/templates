% Source: https://github.com/adibaba/templates
% Author: Adrian Wilke
% Date:   2021-01-28

\documentclass{beamer}
\usetheme{upb}
% Source: https://github.com/adibaba/templates
% Author: Adrian Wilke
% Date:   2021-01-28

% Packages

% inputenc - Accept different input encodings
% https://ctan.org/pkg/inputenc
\usepackage[utf8]{inputenc}

% babel - Multilingual support for Plain TEX or LATEX
% https://ctan.org/pkg/babel
\usepackage[english]{babel}
%\usepackage[german]{babel}

% hyperref – Extensive support for hypertext in LATEX
% https://ctan.org/pkg/hyperref
\usepackage{hyperref}
\hypersetup{
	colorlinks=true,
	linkcolor=templatetext,
	urlcolor=urls
}

% listings – Typeset source code listings using LATEX
% https://ctan.org/pkg/listings
\usepackage{listings}
\lstset{
	basicstyle=\ttfamily,
	%
	keywordstyle=\color{textblue},
	commentstyle=\color{textgray},
	emphstyle=\color{textpurple},
	%
	numbers=left,
	numberstyle=\tiny\color{textgray},
	%
	tabsize=2,
	breaklines=true
}

% appendixnumberbeamer – Manage frame numbering in appendixes in beamer
% https://ctan.org/pkg/appendixnumberbeamer
\usepackage{appendixnumberbeamer} 



% Fonts
% https://tug.org/FontCatalogue/
% https://ctan.org/pkg/free-math-font-survey

% fontenc – Standard package for selecting font encodings
% https://ctan.org/pkg/fontenc
\usepackage[T1]{fontenc}

% Serif font
\usepackage{bera}

% Mono font
\usepackage{nimbusmononarrow}

% Sans-Serif font (fallback)
% tgheros
\usepackage{tgheros}

% Font Roboto Condensed
% https://tug.org/FontCatalogue/robotocondensed/
% https://ctan.org/pkg/roboto
\usepackage[sfdefault,condensed]{roboto}



% Commands

% Text color
\newcommand{\orange}[1]{\textcolor{textorange}{#1}}
\newcommand{\red}[1]{\textcolor{textred}{#1}}
\newcommand{\purple}[1]{\textcolor{textpurple}{#1}}
\newcommand{\blue}[1]{\textcolor{textblue}{#1}}
\newcommand{\darkblue}[1]{\textcolor{textdarkblue}{#1}}
\newcommand{\turquoise}[1]{\textcolor{textturquoise}{#1}}
\newcommand{\green}[1]{\textcolor{textgreen}{#1}}
\newcommand{\gray}[1]{\textcolor{textgray}{#1}}

% Predefined text concepts
\renewcommand{\alert}[1]{\textbf{\red{#1}}}
\renewcommand{\emph}[1]{\blue{#1}}
\newcommand{\good}[1]{\turquoise{#1}}
\newcommand{\bad}[1]{\purple{#1}}
\newcommand{\heading}[1]{\textbf{#1}\\\vspace*{.05cm}}
\newcommand{\term}[1]{\textit{#1}}
\newcommand{\tech}[1]{\texttt{#1}}

% 1st used for slides footer, 2nd for title slide & PDF metadata
\author[A.~Wilke]{Adrian Wilke}
% 1st used for slides footer, 2nd for title slide
\institute[DICE]{DICE Group\\Paderborn University}
% 1st used for slides footer, 2nd for title slide & PDF metadata
\title[Another LaTeX Beamer Template]{Another LaTeX Beamer Template}
% Used for title slide & PDF metadata
\subtitle{}
\date{\today}

\begin{document}

{\usebackgroundtemplate{}
\frame[plain]{\titlepage}}
% Begin counting at second frame
\addtocounter{framenumber}{-1}

\section{Motivation \& Contents}

\begin{frame}{Another LaTeX Beamer Template}
	\begin{columns}[T]
	\column{0.45\textwidth}
		\heading{Motivation}
		After the title slide, typically a table of contents is presented. Make it more interesting by firstly \emph{introducing the problem} you are solving afterwards. That could also be on an own slide. Better than a long text like this is an image or keywords. Almost always.
	\column{0.55\textwidth}
		\heading{Contents}\vspace*{.2cm}
		\hypersetup{linkcolor=textblue}
		\tableofcontents
	\end{columns}
\end{frame}

\section{Quick Start}
\begin{frame}[fragile]{Quick Start}
	Create your first slide:
	\begin{enumerate}
		\item Copy all \emph{*.sty} files into a directory
		\item Copy \emph{packages.tex} into the directory
		\item Create a .tex file and add the code listed below
		\item Generate your slide using LaTeX
	\end{enumerate}
	\lstset{language=TeX, escapechar=\@, keywordstyle=\color{templatetext}, numbers=none, emph={upb,packages,tex}, emphstyle=\color{textblue}}
	\begin{lstlisting}[caption={Minimal Example}]
\documentclass{beamer}
\usetheme{upb}
% Source: https://github.com/adibaba/templates
% Author: Adrian Wilke
% Date:   2021-01-28

% Packages

% inputenc - Accept different input encodings
% https://ctan.org/pkg/inputenc
\usepackage[utf8]{inputenc}

% babel - Multilingual support for Plain TEX or LATEX
% https://ctan.org/pkg/babel
\usepackage[english]{babel}
%\usepackage[german]{babel}

% hyperref – Extensive support for hypertext in LATEX
% https://ctan.org/pkg/hyperref
\usepackage{hyperref}
\hypersetup{
	colorlinks=true,
	linkcolor=templatetext,
	urlcolor=urls
}

% listings – Typeset source code listings using LATEX
% https://ctan.org/pkg/listings
\usepackage{listings}
\lstset{
	basicstyle=\ttfamily,
	%
	keywordstyle=\color{textblue},
	commentstyle=\color{textgray},
	emphstyle=\color{textpurple},
	%
	numbers=left,
	numberstyle=\tiny\color{textgray},
	%
	tabsize=2,
	breaklines=true
}

% appendixnumberbeamer – Manage frame numbering in appendixes in beamer
% https://ctan.org/pkg/appendixnumberbeamer
\usepackage{appendixnumberbeamer} 



% Fonts
% https://tug.org/FontCatalogue/
% https://ctan.org/pkg/free-math-font-survey

% fontenc – Standard package for selecting font encodings
% https://ctan.org/pkg/fontenc
\usepackage[T1]{fontenc}

% Serif font
\usepackage{bera}

% Mono font
\usepackage{nimbusmononarrow}

% Sans-Serif font (fallback)
% tgheros
\usepackage{tgheros}

% Font Roboto Condensed
% https://tug.org/FontCatalogue/robotocondensed/
% https://ctan.org/pkg/roboto
\usepackage[sfdefault,condensed]{roboto}



% Commands

% Text color
\newcommand{\orange}[1]{\textcolor{textorange}{#1}}
\newcommand{\red}[1]{\textcolor{textred}{#1}}
\newcommand{\purple}[1]{\textcolor{textpurple}{#1}}
\newcommand{\blue}[1]{\textcolor{textblue}{#1}}
\newcommand{\darkblue}[1]{\textcolor{textdarkblue}{#1}}
\newcommand{\turquoise}[1]{\textcolor{textturquoise}{#1}}
\newcommand{\green}[1]{\textcolor{textgreen}{#1}}
\newcommand{\gray}[1]{\textcolor{textgray}{#1}}

% Predefined text concepts
\renewcommand{\alert}[1]{\textbf{\red{#1}}}
\renewcommand{\emph}[1]{\blue{#1}}
\newcommand{\good}[1]{\turquoise{#1}}
\newcommand{\bad}[1]{\purple{#1}}
\newcommand{\heading}[1]{\textbf{#1}\\\vspace*{.05cm}}
\newcommand{\term}[1]{\textit{#1}}
\newcommand{\tech}[1]{\texttt{#1}}
\begin{document}
\begin{frame} Hello World @\textbackslash@end{frame}
@\textbackslash@end{document}
	\end{lstlisting}
\end{frame}

\section{Text Formatting}
\subsection{Predefined Styles}
\begin{frame}{Text Formatting}{Predefined Styles}
	\begin{itemize}
		\item You could \emph{emphasize} important parts \\(Maybe distinguish between \bad{problems} and \good{solutions})
		\item Highlight ``\term{predefined terms}'' like brands and\\ \tech{TechnicalTerms} like software components
		\item Use alert to display \alert{warnings}
		\item Use the url command for links (\url{https://upb.de/}) or href (\href{https://github.com/adibaba/templates}{template})
	\end{itemize}
\end{frame}

\subsection{Additional Commands}
\begin{frame}{Text Formatting}{Additional Commands}
	Use combinations for other concepts:
	\begin{itemize}
		\item Text styles: \textbf{bold}, \textit{italic}, \underline{underlined}, \textsc{small caps}
		\item Font families: \texttt{monospaced}, \textsf{sans serif}, \textrm{roman}
		\item Text colors: \orange{orange}, \red{red}, \purple{purple}, \darkblue{darkblue}, \blue{blue}, \turquoise{turquoise}, \green{green}, \gray{gray}
		\item Text sizes: {\tiny tiny}, {\scriptsize scriptsize}, {\footnotesize footnotesize}, {\small small}, {\normalsize normalsize}, {\large large}, {\Large Large}, {\LARGE LARGE}, {\huge huge}, {\Huge Huge}
	\end{itemize}
\end{frame}

\section{Code Listings \& Frame Arguments}
\begin{frame}[fragile]{Code Listings \& Frame Arguments}
	Use these arguments to configure frames:
	\setbeamercolor{local structure}{fg=textpurple}
	\begin{description}
		\item[\tech{fragile}\hfill]
		Specially interpreted contents, e.g. for listings
		\item[\tech{plain}\hfill]
		No headlines, footlines, sidebars; e.g. for large images\\
		\gray{To also remove background images use:
		\texttt{\{\textbackslash usebackgroundtemplate\{\}[...]\}}}
		\item[\tech{squeeze}\hfill]
		Squeezes vertical spaces, e.g. for long contents
		\item[\tech{shrink}\hfill]
		Shrinks frame, e.g. for long contents
	\end{description}
	\lstset{language=TeX, escapechar=\@, morekeywords={begin,end,frame}, emph={fragile}}
	\begin{lstlisting}[caption={Frame Options}]
\begin{frame}[fragile]{Code Listings \& Frame Arguments}
	% [...]
@\textbackslash@end{frame}
	\end{lstlisting}
\end{frame}

\section{Mathematics \& Miscellaneous}
\begin{frame}{Mathematics \& Miscellaneous}
	\begin{itemize}
		\item Math\footnote[frame]{This is a footnote also working in columns}: $5^{2}=3^{2}+4^{2}$
		\item Equations:
	\end{itemize}
	\begin{equation}
		\sum_{n = 1}^{\infty} \frac{1}{n} = 1 + \frac{1}{2} + \frac{1}{3} + \frac{1}{4} + \frac{1}{5} + \dots 
	\end{equation}
\end{frame}

\section{Blocks}
\begin{frame}[shrink]{Blocks}
	\begin{block}{This is a Block}
		\begin{itemize}
			\item This is an item
		\end{itemize}
		\begin{enumerate}
			\item This is enumeration item
		\end{enumerate}
	\end{block}
	\begin{exampleblock}{This is an Example Block}
		\begin{itemize}
			\item This is an item
		\end{itemize}
		\begin{enumerate}
			\item This is enumeration item
		\end{enumerate}
	\end{exampleblock}
	\begin{alertblock}{This is an Alert Block}
		\begin{itemize}
			\item This is an item
		\end{itemize}
		\begin{enumerate}
			\item This is enumeration item
		\end{enumerate}
	\end{alertblock}
\end{frame}

\section{Graphs}
\begin{frame}{Graphs}
	\centering
	\begin{tikzpicture}[node distance=1cm and 2cm] 
		\node[resource] (pb) {Subject}; 
		\node[literal, right=of pb] (code) {Object};
		\draw[->] (pb) -- node[above]{Predicate} (code);
	\end{tikzpicture}
\end{frame}

\section{Appendix \& References}
\appendix 

\begin{frame}{Appendix}{Predefined Text Colors}
	\begin{itemize}
		\item {\color{textorange}\rule{.7cm}{.4cm} textorange}
		\item {\color{textred}\rule{.7cm}{.4cm} textred}
		\item {\color{textpurple}\rule{.7cm}{.4cm} textpurple}
		\item {\color{textdarkblue}\rule{.7cm}{.4cm} textdarkblue}
		\item {\color{textblue}\rule{.7cm}{.4cm} textblue}
		\item {\color{textturquoise}\rule{.7cm}{.4cm} textturquoise}
		\item {\color{textgreen}\rule{.7cm}{.4cm} textgreen}
		\item {\color{textgray}\rule{.7cm}{.4cm} textgray}
	\end{itemize}
\end{frame}

\begin{frame}{Appendix}{Predefined UPB Colors}
	\begin{itemize}
		\item Some of the official UPB colors, some are directly used in this template
		\item UPB primary:
		      {\color{upbblue}\rule{.7cm}{.4cm} upbblue},
		      {\color{upbcyan}\rule{.7cm}{.4cm} upbcyan},
		      {\color{upbgraylight}\rule{.7cm}{.4cm} upbgraylight},
		      {\color{upbgraydark}\rule{.7cm}{.4cm} upbgraydark}
		\item UPB secondary:
		      {\color{upb2magenta}\rule{.7cm}{.4cm} upb2magenta},
		      {\color{upb2blue}\rule{.7cm}{.4cm} upb2blue},
		      {\color{upb2green}\rule{.7cm}{.4cm} upb2green},
		      {\color{upb2orange}\rule{.7cm}{.4cm} upb2orange},
		      {\color{upb2purple}\rule{.7cm}{.4cm} upb2purple}
		\item UPB active:
		      {\color{upbayellow}\rule{.7cm}{.4cm} upbayellow},
		      {\color{upbared}\rule{.7cm}{.4cm} upbared},
		      {\color{upbapurple}\rule{.7cm}{.4cm} upbapurple},
		      {\color{upbaturquoise}\rule{.7cm}{.4cm} upbaturquoise}
		\item UPB color source:
		      \href{https://www.uni-paderborn.de/en/university/press-communications-marketing/design-vorgaben-templates/colors}{The Colors of UPB} (2018) and
		      \href{https://www.uni-paderborn.de/en/university/press-communications-marketing/corporate-design-manual}{The Corporate Design Manual} (2016)
	\end{itemize}
\end{frame}

\begin{frame}[t,allowframebreaks]{References}
	\scriptsize
	\nocite{*}
	\bibliographystyle{ieeetr}
	\bibliography{bibliography}
\end{frame}

\end{document}