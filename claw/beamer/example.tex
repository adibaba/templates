% Source: https://github.com/adibaba/templates
% Author: Adrian Wilke
% Date:   2021-03-11

\documentclass{beamer}

% Alternative to \usetheme{claw}
\usepackage{template/beamercolorthemeclaw}
\usepackage{template/beamerfontthemeclaw}
\usepackage{template/beamerinnerthemeclaw}
\usepackage{template/beamerouterthemeclaw}

% Source: https://github.com/adibaba/templates
% Author: Adrian Wilke
% Date:   2021-03-11



% Configuration

% Path for images
\graphicspath{{./images/}}

% OPTIONAL: Background image with logos
%\usebackgroundtemplate{\includegraphics[width=\paperwidth]{background}}
% Reduce max. frame title width (no logo overlap)
%\renewcommand{\frametitlewidth}{.8\textwidth}

% OPTIONAL: Replace blue with specific blue
%\colorlet{textblue}{textbluespecific}
%\colorlet{elementblue}{elementbluespecific}
%\colorlet{lightblue}{lightbluespecific}
%\colorlet{backgroundblue}{backgroundbluespecific}



% Packages

% inputenc - Accept different input encodings
% https://ctan.org/pkg/inputenc
\usepackage[utf8x]{inputenc}

% babel - Multilingual support for Plain TEX or LATEX
% https://ctan.org/pkg/babel
\usepackage[english]{babel}
%\usepackage[german]{babel}

% hyperref – Extensive support for hypertext in LATEX
% https://ctan.org/pkg/hyperref
\usepackage{hyperref}
\hypersetup{
	colorlinks=true,
	linkcolor=footertext,
	urlcolor=urls
}

% appendixnumberbeamer – Manage frame numbering in appendixes in beamer
% https://ctan.org/pkg/appendixnumberbeamer
\usepackage{appendixnumberbeamer} 



% Fonts
% https://tug.org/FontCatalogue/
% https://ctan.org/pkg/free-math-font-survey

% fontenc – Standard package for selecting font encodings
% https://ctan.org/pkg/fontenc
\usepackage[T1]{fontenc}

% Serif font
\usepackage{bera}

% Mono font
\usepackage{nimbusmononarrow}

% Sans-Serif font (fallback)
% tgheros
\usepackage{tgheros}

% Font Roboto Condensed
% https://tug.org/FontCatalogue/robotocondensed/
% https://ctan.org/pkg/roboto
\usepackage[sfdefault,condensed]{roboto}

% Alternative: Arial Narrow
% Fontspec requires XeTeX or LuaTeX, not PDFLaTeX
%\usepackage{fontspec}
%\setsansfont[Ligatures=TeX]{Arial Narrow}



% Optional packages
% Can be removed if not used

% pgf – Create PostScript and PDF graphics in TEX
% https://www.ctan.org/pkg/pgf
\usepackage{tikz}
\usetikzlibrary{shapes,positioning}
\usetikzlibrary{arrows.meta,arrows}
\tikzset{resource/.style={draw,rectangle,fill=lightblue,rounded corners,inner sep=5pt} }
\tikzset{literal/.style={draw,rectangle,fill=lightgreen,inner sep=5pt}}
\tikzset{>=triangle 45}
\tikzset{every picture/.style=thick}
\tikzstyle{every node}=[font=\small]

% listings – Typeset source code listings using LATEX
% https://ctan.org/pkg/listings
\usepackage{listings}
\lstset{
	basicstyle=\small\ttfamily,
	%
	keywordstyle=\color{textblue},
	commentstyle=\color{textgray},
	emphstyle=\color{textpurple},
	%
	numbers=left,
	numbersep=7pt,
	numberstyle=\scriptsize\color{textgray}\ttfamily,
	%
	tabsize=2,
	breaklines=true
}



% Beamer settings

% Disable navigation symbols
\setbeamertemplate{navigation symbols}{}

% Move title page contents down
\addtobeamertemplate{title page}{\vspace{2\baselineskip}}{}

% OPTIONAL: Alternative footer
%\useoutertheme{infolines}
%\setbeamertemplate{headline}{}
%\setbeamerfont{title in head/foot}{series=\mdseries}

% OPTIONAL: Alternative header/footer colors
%\setbeamercolor{palette primary}{bg=lightgray}
%\setbeamercolor{palette secondary}{bg=lightgray}
%\setbeamercolor{palette tertiary}{bg=lightgray}
%\setbeamercolor{frametitle}{bg=lightgray}



% Commands

% Text color
\newcommand{\bluedark}[1]{\textcolor{textbluedark}{#1}}
\newcommand{\blue}[1]{\textcolor{textblue}{#1}}
\newcommand{\gray}[1]{\textcolor{textgray}{#1}}
%
\newcommand{\magenta}[1]{\textcolor{textmagenta}{#1}}
\newcommand{\orange}[1]{\textcolor{textorange}{#1}}
\newcommand{\purple}[1]{\textcolor{textpurple}{#1}}
%
\newcommand{\red}[1]{\textcolor{textred}{#1}}
\newcommand{\turquoise}[1]{\textcolor{textturquoise}{#1}}
\newcommand{\green}[1]{\textcolor{textgreen}{#1}}

% Predefined text concepts
\renewcommand{\alert}[1]{\textbf{\red{#1}}}
\renewcommand{\emph}[1]{\blue{#1}}
\newcommand{\good}[1]{\turquoise{#1}}
\newcommand{\bad}[1]{\purple{#1}}
\newcommand{\heading}[1]{\textbf{#1}\\\vspace*{.05cm}}
\newcommand{\tech}[1]{\gray{\texttt{#1}}}
\newcommand{\term}[1]{\gray{\textit{#1}}}

% OPTIONAL: Example blocks in green instead of turquoise
%\setbeamercolor{block title example}{bg=lightgreen,fg=templatetext}
%\setbeamercolor{block body example}{bg=backgroundgreen}
% OPTIONAL: Positive highlights in green instead of turquoise
%\renewcommand{\good}[1]{\green{#1}}

% OPTIONAL: Alert blocks in red instead of purple
%\setbeamercolor{block title alerted}{bg=lightred,fg=templatetext}
%\setbeamercolor{block body alerted}{bg=backgroundred}
% OPTIONAL: Negative highlights in red instead of purple
%\renewcommand{\bad}[1]{\red{#1}}

% OPTIONAL: Alert blocks in magenta instead of purple
%\setbeamercolor{block title alerted}{bg=lightmagenta,fg=templatetext}
%\setbeamercolor{block body alerted}{bg=backgroundmagenta}
% OPTIONAL: Negative highlights in magenta instead of purple
%\renewcommand{\bad}[1]{\magenta{#1}}




% 1st used for slides footer, 2nd for title slide & PDF metadata
\title[Claw LaTeX Beamer Template]{Claw LaTeX Beamer Template}
% Used for title slide & PDF metadata
\subtitle{}

\date{\today}
% Date with conference name
%\date[\today]{Conference on Fabulous Presentations, 2021}

% One author
% 1st used for slides footer, 2nd for title slide & PDF metadata
\author[A.~Wilke]{Adrian Wilke}

% 1st used for slides footer, 2nd for title slide
\institute[DICE]{DICE Group\\Paderborn University}
% Institute with logo
%\institute[DICE]{\includegraphics[height=1.1cm]{LOGO}\\Data Science Group}

% Two authors from different institutes
% Source: https://www.overleaf.com/learn/latex/Beamer#The_title_page
%\author[\underline{Wilke}, Ngonga]{\underline{Adrian~Wilke}\inst{1} \and Axel~Ngonga\inst{2}}
%\institute[]
%{
%	\inst{1}%
%	DICE Group\\
%	Department of Computer Science\\
%	Paderborn University, Germany
%	\and
%	\inst{2}%
%	DICE Group\\
%	Department of Computer Science\\
%	Paderborn University, Germany
%}

% One logo on title page
%\titlegraphic{\includegraphics[height=1cm]{LOGO}}

% Two logos on title page
%\titlegraphic{
%	\includegraphics[height=1cm]{LOGO}%
%	\hspace*{2cm}~%
%	\includegraphics[height=1cm]{LOGO2}%
%}



\begin{document}

% Title page without background
{\usebackgroundtemplate{}
\frame[plain]{\titlepage}}

% Title page with background
%{\usebackgroundtemplate{\includegraphics[width=\paperwidth]{background-title}}
%\frame[plain]{\titlepage}}

% Begin counting at second frame
\addtocounter{framenumber}{-1}

\section{Motivation \& Contents}

\begin{frame}{Claw LaTeX Beamer Template}
	\begin{columns}[T]
	\column{0.45\textwidth}
		\heading{Motivation}
		After the title slide, typically a table of contents is presented. Make it more interesting by firstly \emph{introducing the problem} you are solving afterwards. That could also be on an own slide. Better than a long text like this is an image or keywords. Almost always.
	\column{0.55\textwidth}
		\heading{Contents}\vspace*{.2cm}
		\hypersetup{linkcolor=textblue}
		\tableofcontents
	\end{columns}
\end{frame}

\section{Quick Start}
\begin{frame}[fragile]{Quick Start}
	Create your first slide:
	\begin{enumerate}
		\item Copy all \emph{*.sty} files into a directory
		\item Copy \emph{packages.tex} into the directory
		\item Create a .tex file and add the code listed below
		\item Generate your slide using LaTeX
	\end{enumerate}
	\lstset{language=TeX, escapechar=\@, keywordstyle=\color{templatetext}, numbers=none, emph={claw,packages,tex}, emphstyle=\color{textblue}}
	\begin{lstlisting}[caption={Minimal Example}]
\documentclass{beamer}
\usetheme{claw}
% Source: https://github.com/adibaba/templates
% Author: Adrian Wilke
% Date:   2021-03-11



% Configuration

% Path for images
\graphicspath{{./images/}}

% OPTIONAL: Background image with logos
%\usebackgroundtemplate{\includegraphics[width=\paperwidth]{background}}
% Reduce max. frame title width (no logo overlap)
%\renewcommand{\frametitlewidth}{.8\textwidth}

% OPTIONAL: Replace blue with specific blue
%\colorlet{textblue}{textbluespecific}
%\colorlet{elementblue}{elementbluespecific}
%\colorlet{lightblue}{lightbluespecific}
%\colorlet{backgroundblue}{backgroundbluespecific}



% Packages

% inputenc - Accept different input encodings
% https://ctan.org/pkg/inputenc
\usepackage[utf8x]{inputenc}

% babel - Multilingual support for Plain TEX or LATEX
% https://ctan.org/pkg/babel
\usepackage[english]{babel}
%\usepackage[german]{babel}

% hyperref – Extensive support for hypertext in LATEX
% https://ctan.org/pkg/hyperref
\usepackage{hyperref}
\hypersetup{
	colorlinks=true,
	linkcolor=footertext,
	urlcolor=urls
}

% appendixnumberbeamer – Manage frame numbering in appendixes in beamer
% https://ctan.org/pkg/appendixnumberbeamer
\usepackage{appendixnumberbeamer} 



% Fonts
% https://tug.org/FontCatalogue/
% https://ctan.org/pkg/free-math-font-survey

% fontenc – Standard package for selecting font encodings
% https://ctan.org/pkg/fontenc
\usepackage[T1]{fontenc}

% Serif font
\usepackage{bera}

% Mono font
\usepackage{nimbusmononarrow}

% Sans-Serif font (fallback)
% tgheros
\usepackage{tgheros}

% Font Roboto Condensed
% https://tug.org/FontCatalogue/robotocondensed/
% https://ctan.org/pkg/roboto
\usepackage[sfdefault,condensed]{roboto}

% Alternative: Arial Narrow
% Fontspec requires XeTeX or LuaTeX, not PDFLaTeX
%\usepackage{fontspec}
%\setsansfont[Ligatures=TeX]{Arial Narrow}



% Optional packages
% Can be removed if not used

% pgf – Create PostScript and PDF graphics in TEX
% https://www.ctan.org/pkg/pgf
\usepackage{tikz}
\usetikzlibrary{shapes,positioning}
\usetikzlibrary{arrows.meta,arrows}
\tikzset{resource/.style={draw,rectangle,fill=lightblue,rounded corners,inner sep=5pt} }
\tikzset{literal/.style={draw,rectangle,fill=lightgreen,inner sep=5pt}}
\tikzset{>=triangle 45}
\tikzset{every picture/.style=thick}
\tikzstyle{every node}=[font=\small]

% listings – Typeset source code listings using LATEX
% https://ctan.org/pkg/listings
\usepackage{listings}
\lstset{
	basicstyle=\small\ttfamily,
	%
	keywordstyle=\color{textblue},
	commentstyle=\color{textgray},
	emphstyle=\color{textpurple},
	%
	numbers=left,
	numbersep=7pt,
	numberstyle=\scriptsize\color{textgray}\ttfamily,
	%
	tabsize=2,
	breaklines=true
}



% Beamer settings

% Disable navigation symbols
\setbeamertemplate{navigation symbols}{}

% Move title page contents down
\addtobeamertemplate{title page}{\vspace{2\baselineskip}}{}

% OPTIONAL: Alternative footer
%\useoutertheme{infolines}
%\setbeamertemplate{headline}{}
%\setbeamerfont{title in head/foot}{series=\mdseries}

% OPTIONAL: Alternative header/footer colors
%\setbeamercolor{palette primary}{bg=lightgray}
%\setbeamercolor{palette secondary}{bg=lightgray}
%\setbeamercolor{palette tertiary}{bg=lightgray}
%\setbeamercolor{frametitle}{bg=lightgray}



% Commands

% Text color
\newcommand{\bluedark}[1]{\textcolor{textbluedark}{#1}}
\newcommand{\blue}[1]{\textcolor{textblue}{#1}}
\newcommand{\gray}[1]{\textcolor{textgray}{#1}}
%
\newcommand{\magenta}[1]{\textcolor{textmagenta}{#1}}
\newcommand{\orange}[1]{\textcolor{textorange}{#1}}
\newcommand{\purple}[1]{\textcolor{textpurple}{#1}}
%
\newcommand{\red}[1]{\textcolor{textred}{#1}}
\newcommand{\turquoise}[1]{\textcolor{textturquoise}{#1}}
\newcommand{\green}[1]{\textcolor{textgreen}{#1}}

% Predefined text concepts
\renewcommand{\alert}[1]{\textbf{\red{#1}}}
\renewcommand{\emph}[1]{\blue{#1}}
\newcommand{\good}[1]{\turquoise{#1}}
\newcommand{\bad}[1]{\purple{#1}}
\newcommand{\heading}[1]{\textbf{#1}\\\vspace*{.05cm}}
\newcommand{\tech}[1]{\gray{\texttt{#1}}}
\newcommand{\term}[1]{\gray{\textit{#1}}}

% OPTIONAL: Example blocks in green instead of turquoise
%\setbeamercolor{block title example}{bg=lightgreen,fg=templatetext}
%\setbeamercolor{block body example}{bg=backgroundgreen}
% OPTIONAL: Positive highlights in green instead of turquoise
%\renewcommand{\good}[1]{\green{#1}}

% OPTIONAL: Alert blocks in red instead of purple
%\setbeamercolor{block title alerted}{bg=lightred,fg=templatetext}
%\setbeamercolor{block body alerted}{bg=backgroundred}
% OPTIONAL: Negative highlights in red instead of purple
%\renewcommand{\bad}[1]{\red{#1}}

% OPTIONAL: Alert blocks in magenta instead of purple
%\setbeamercolor{block title alerted}{bg=lightmagenta,fg=templatetext}
%\setbeamercolor{block body alerted}{bg=backgroundmagenta}
% OPTIONAL: Negative highlights in magenta instead of purple
%\renewcommand{\bad}[1]{\magenta{#1}}

\begin{document}
\begin{frame} Hello World @\textbackslash@end{frame}
@\textbackslash@end{document}
	\end{lstlisting}
\end{frame}

\section{Text Formatting}
\subsection{Predefined Styles}
\begin{frame}{Text Formatting}{Predefined Styles}
	\begin{itemize}
		\item You could \emph{emphasize} important parts \\(Maybe distinguish between \bad{problems} and \good{solutions})
		\item Use alert to display \alert{warnings}
		\item Use the url command (\url{https://dice-research.org/}) or the href command (\href{https://dice-research.org/}{DICE}) for links
		\item Highlight ``\term{predefined terms}'' like brands and\\ \tech{TechnicalTerms} like software components
	\end{itemize}
\end{frame}

\subsection{Additional Commands}
\begin{frame}{Text Formatting}{Additional Commands}
	Use combinations for other concepts:
	\begin{itemize}
		\item Text styles: \textbf{bold}, \textit{italic}, \underline{underlined}, \textsc{small caps}
		\item Font families: \texttt{monospaced}, \textsf{sans serif}, \textrm{roman}
		\item Text colors:
			\bluedark{bluedark},
			\gray{gray},
			\magenta{magenta},
			\blue{blue},
			\orange{orange},
			\purple{purple},
			\red{red},
			\turquoise{turquoise},
			\green{green}
		\item Text sizes: {\tiny tiny}, {\scriptsize scriptsize}, {\footnotesize footnotesize}, {\small small}, {\normalsize normalsize}, {\large large}, {\Large Large}, {\LARGE LARGE}, {\huge huge}, {\Huge Huge}
	\end{itemize}
\end{frame}

\section{Code Listings \& Frame Arguments}
\begin{frame}[fragile]{Code Listings \& Frame Arguments}
	Use these arguments to configure frames:
	\setbeamercolor{local structure}{fg=textpurple}
	\begin{description}
		\item[\texttt{fragile}\hfill]
		Specially interpreted contents, e.g. for listings
		\item[\texttt{plain}\hfill]
		No headlines, footlines, sidebars; e.g. for large images\\
		\gray{To also remove background images use:
		\texttt{\{\textbackslash usebackgroundtemplate\{\}[...]\}}}
		\item[\texttt{squeeze}\hfill]
		Squeezes vertical spaces, e.g. for long contents
		\item[\texttt{shrink}\hfill]
		Shrinks frame, e.g. for long contents
	\end{description}
	\lstset{language=TeX, escapechar=\@, morekeywords={begin,end,frame}, emph={fragile}}
	\begin{lstlisting}[caption={Frame Options}]
\begin{frame}[fragile]{Code Listings \& Frame Arguments}
	% [...]
@\textbackslash@end{frame}
	\end{lstlisting}
\end{frame}

\section{Mathematics \& Miscellaneous}
\begin{frame}{Mathematics \& Miscellaneous}
	\begin{itemize}
		\item Math\footnote[frame]{This is a footnote also working in columns}: $5^{2}=3^{2}+4^{2}$
		\item Equations:
	\end{itemize}
	\begin{equation}
		\sum_{n = 1}^{\infty} \frac{1}{n} = 1 + \frac{1}{2} + \frac{1}{3} + \frac{1}{4} + \frac{1}{5} + \dots 
	\end{equation}
\end{frame}

\section{Blocks}
\begin{frame}[shrink]{Blocks}
	\begin{block}{This is a Block}
		\begin{itemize}
			\item This is an item
		\end{itemize}
		\begin{enumerate}
			\item This is enumeration item
		\end{enumerate}
	\end{block}
	\begin{exampleblock}{This is an Example Block}
		\begin{itemize}
			\item This is an item
		\end{itemize}
		\begin{enumerate}
			\item This is enumeration item
		\end{enumerate}
	\end{exampleblock}
	\begin{alertblock}{This is an Alert Block}
		\begin{itemize}
			\item This is an item
		\end{itemize}
		\begin{enumerate}
			\item This is enumeration item
		\end{enumerate}
	\end{alertblock}
\end{frame}

% Examples for alternative colors. Configure this in packages.tex, not here.
\setbeamercolor{block title}{bg=lightorange,fg=templatetext}
\setbeamercolor{block body}{bg=backgroundorange}
\setbeamercolor{block title example}{bg=lightgreen,fg=templatetext}
\setbeamercolor{block body example}{bg=backgroundgreen}
\setbeamercolor{block title alerted}{bg=lightred,fg=templatetext}
\setbeamercolor{block body alerted}{bg=backgroundred}
\begin{frame}[shrink]{Blocks}
	\begin{block}{This is a Block}
		\begin{itemize}
			\item This is an item
		\end{itemize}
		\begin{enumerate}
			\item This is enumeration item
		\end{enumerate}
	\end{block}
	\begin{exampleblock}{This is an Example Block}
		\begin{itemize}
			\item This is an item
		\end{itemize}
		\begin{enumerate}
			\item This is enumeration item
		\end{enumerate}
	\end{exampleblock}
	\begin{alertblock}{This is an Alert Block}
		\begin{itemize}
			\item This is an item
		\end{itemize}
		\begin{enumerate}
			\item This is enumeration item
		\end{enumerate}
	\end{alertblock}
\end{frame}

\section{Graphs}
\begin{frame}{Graphs}
	\centering
	\begin{tikzpicture}[node distance=1cm and 2cm] 
		\node[resource] (pb) {Subject}; 
		\node[literal, right=of pb] (code) {Object};
		\draw[->] (pb) -- node[above]{Predicate} (code);
	\end{tikzpicture}
\end{frame}

\section{Appendix \& References}
\appendix 

\begin{frame}{Appendix}{Predefined Base Colors}
	\begin{itemize}
		\item {\color{primarybluedark}\rule{.7cm}{.4cm} primarybluedark}
		\item {\color{primarybluelight}\rule{.7cm}{.4cm} primarybluelight}
		\item {\color{primarygraylight}\rule{.7cm}{.4cm} primarygraylight}
		\item {\color{primarygraydark}\rule{.7cm}{.4cm} primarygraydark}
		\item {\color{secondarymagenta}\rule{.7cm}{.4cm} secondarymagenta}
		\item {\color{secondaryblue}\rule{.7cm}{.4cm} secondaryblue}
		\item {\color{secondarygreen}\rule{.7cm}{.4cm} secondarygreen}
		\item {\color{secondaryorange}\rule{.7cm}{.4cm} secondaryorange}
		\item {\color{secondarypurple}\rule{.7cm}{.4cm} secondarypurple}
		\item {\color{activeyellow}\rule{.7cm}{.4cm} activeyellow}
		\item {\color{activered}\rule{.7cm}{.4cm} activered}
		\item {\color{activeturquoise}\rule{.7cm}{.4cm} activeturquoise}
		\item {\color{activegreen}\rule{.7cm}{.4cm} activegreen}
		\item {\color{specificblue}\rule{.7cm}{.4cm} specificblue}
	\end{itemize}
\end{frame}

\begin{frame}{Appendix}{Predefined Text Colors}
	\begin{itemize}
		\item {\color{textbluedark}\rule{.7cm}{.4cm} textdarkblue}
		\item {\color{textgray}\rule{.7cm}{.4cm} textgray}
		\item {\color{textmagenta}\rule{.7cm}{.4cm} textmagenta}
		\item {\color{textblue}\rule{.7cm}{.4cm} textblue}
		\item {\color{textorange}\rule{.7cm}{.4cm} textorange}
		\item {\color{textpurple}\rule{.7cm}{.4cm} textpurple}
		\item {\color{textred}\rule{.7cm}{.4cm} textred}
		\item {\color{textturquoise}\rule{.7cm}{.4cm} textturquoise}
		\item {\color{textgreen}\rule{.7cm}{.4cm} textgreen}
		\item {\color{textbluespecific}\rule{.7cm}{.4cm} textbluespecific}
	\end{itemize}
\end{frame}

\begin{frame}{Appendix}{Predefined Element Colors}
	\begin{itemize}
		\item {\color{elementgray}\rule{.7cm}{.4cm} elementgray}
		\item {\color{elementmagenta}\rule{.7cm}{.4cm} elementmagenta}
		\item {\color{elementblue}\rule{.7cm}{.4cm} elementblue}
		\item {\color{elementorange}\rule{.7cm}{.4cm} elementorange}
		\item {\color{elementpurple}\rule{.7cm}{.4cm} elementpurple}
		\item {\color{elementyellow}\rule{.7cm}{.4cm} elementyellow}
		\item {\color{elementred}\rule{.7cm}{.4cm} elementred}
		\item {\color{elementturquoise}\rule{.7cm}{.4cm} elementturquoise}
		\item {\color{elementgreen}\rule{.7cm}{.4cm} elementgreen}
		\item {\color{elementbluespecific}\rule{.7cm}{.4cm} elementbluespecific}
	\end{itemize}
\end{frame}

\begin{frame}{Appendix}{Predefined Light Colors}
	\begin{itemize}
		\item {\color{lightgray}\rule{.7cm}{.4cm} lightgray}
		\item {\color{lightmagenta}\rule{.7cm}{.4cm} lightmagenta}
		\item {\color{lightblue}\rule{.7cm}{.4cm} lightblue}
		\item {\color{lightorange}\rule{.7cm}{.4cm} lightorange}
		\item {\color{lightpurple}\rule{.7cm}{.4cm} lightpurple}
		\item {\color{lightyellow}\rule{.7cm}{.4cm} lightyellow}
		\item {\color{lightred}\rule{.7cm}{.4cm} lightred}
		\item {\color{lightturquoise}\rule{.7cm}{.4cm} lightturquoise}
		\item {\color{lightgreen}\rule{.7cm}{.4cm} lightgreen}
		\item {\color{lightbluespecific}\rule{.7cm}{.4cm} lightbluespecific}
	\end{itemize}
\end{frame}

\begin{frame}{Appendix}{Predefined Background Colors}
	\begin{itemize}
		\item {\color{backgroundgray}\rule{.7cm}{.4cm}} backgroundgray
		\item {\color{backgroundmagenta}\rule{.7cm}{.4cm}} backgroundmagenta
		\item {\color{backgroundblue}\rule{.7cm}{.4cm}} backgroundblue
		\item {\color{backgroundorange}\rule{.7cm}{.4cm}} backgroundorange
		\item {\color{backgroundpurple}\rule{.7cm}{.4cm}} backgroundpurple
		\item {\color{backgroundyellow}\rule{.7cm}{.4cm}} backgroundyellow
		\item {\color{backgroundred}\rule{.7cm}{.4cm}} backgroundred
		\item {\color{backgroundturquoise}\rule{.7cm}{.4cm}} backgroundturquoise
		\item {\color{backgroundgreen}\rule{.7cm}{.4cm}} backgroundgreen
		\item {\color{backgroundbluespecific}\rule{.7cm}{.4cm}} backgroundbluespecific
	\end{itemize}
\end{frame}

\begin{frame}[t,allowframebreaks]{References}
	\scriptsize
	\nocite{*}
	\bibliographystyle{ieeetr}
	\bibliography{bibliography}
\end{frame}

\end{document}